\documentclass{article}
\usepackage{amsmath}
\usepackage{hyperref}
\usepackage{graphicx}  % Add this line to include the graphicx package

\title{Hledání dostatečného počtu tahů k zamíchání Rubikovy kostky}
\author{Viktor Číhal}
\date{}


\begin{document}
\maketitle
\section*{Úvod}
V této práci jsem se rozhodl zkoumat vliv počtu tahů při míchání na dosažené skóre svého řešiče.
Řešič má určité podobnosti s lidskými řešiteli, a proto by výsledky mohly být použitelné i pro ně.
Zdrojový kód a notebook s experimenty se dá nalézt na \url{https://github.com/Reblexis/rubik-solver}.

\section*{Popis řešiče}
\subsection*{Algoritmus}
Řešič funguje na základě opakovaného prohledávání stavového prostoru metodou DFS
s dodatečným přidáváním náhodných tahů, pokud není nalezen zlepšující stav kostky.
K rozhodnutí o tom, zda je nalezen zlepšující stav, se používá metrika, počítající
správné kostičky v kostce. Hledání je rozděleno do 2 fází. V první fázi se hledá
řešení, které dostane kostku do podgrupy G1 a ve druhé fázi se pomocí menšího počtu
různých tahů (a tedy i větší hloubky) hledá již úplné řešení.\\
Řešič tedy většinou nenalezne nejkratší řešení.

\subsection*{Parametry}
Při spouštění lze nastavovat tyto parametry:
\begin{itemize}
    \item Počet tahů zamíchání kostky ($n$)
    \item Časový limit pro hledání řešení (v milisekundách) ($t$)
    \item Maximální hloubka hledání v první fázi ($d1$)
    \item Maximální hloubka hledání ve druhé fázi ($d2$)
\end{itemize}

Více informací se nachází v \href{https://github.com/Reblexis/rubik-solver/blob/main/README.md}{README.md} souboru.

\section*{Vliv počtu tahů při míchání na dosažené skóre}
Rozhodl jsem se prozkoumat tento vztah, neboť se hodí vědět kolik stačí tahů při míchání, aby byla kostka
`dostatečně náhodná'. \\
\subsection*{Graf závislosti}
Skóre je počítáno jako počet správně umístěných kostiček v kostce (tedy na správné pozici a zároveň správně otočených).
Pro každou testovanou hodnotu $n$ spustíme 100-krát řešič a ostatní parametry zafixujeme na
$t = 100$, $d1 = 4$ a $d2 = 5$.\\
Algoritmus sice řeší kostku jiným způsobem než člověk, ale je založený na podobném principu 
postupného přesouvání kostiček na správné pozice.

\begin{figure}[h]
    \centering
    \includegraphics[width=0.8\textwidth]{scramble_moves_influence.png}
    \caption{Vliv počtu tahů při míchání na dosažené skóre}
    \label{fig:scramble_moves_influence}
\end{figure}

Z grafu můžem vypozorovat, že 15 míchacích tahů nejspíš nestačí a od 20 míchacích
tahů by již kostka mohla být dostatečně náhodná pro lidské řešitele.\\
Pojďme se ale ujistit, že 15 tahů opravdu není dostatečné.

\subsection*{Test normality distribuce}
Nejprve provedeme test, zda mají distribuce skóre normální rozdělení.
Provedeme tedy Shapiro-Wilkův test s nulovou hypotézou, že data pocházejí z normálního rozdělení.\\
Spustíme 100-krát řešič jednotlivě pro hodnoty $n = 15$, $n = 20$ a $n = 50$ (což by již mělo být dostatečně náhodné)
a zafixujeme $t = 100$, $d1 = 4$ a $d2 = 5$.\\
Z testu nám vyšla p-hodnota výrazně menší než 0.05, což znamená, že zamítáme nulovou hypotézu a 
můžeme říct, že distribuce skóre není normální.\\

\begin{figure}[h]
    \centering
    \includegraphics[width=0.8\textwidth]{histogram_scores.png}
    \caption{Histogram skóre pro $n = 15$, $n = 20$ a $n = 50$}
    \label{fig:histogram_scores}
\end{figure}

\subsection*{KS test}
Jelikož distribuce není normální, nemůžeme použít t-test a místo toho tedy provedeme
dvouvýběrový KS test s nulovou hypotézou, že distribuce jsou stejné.\\
Pro pár $n = 15$ a $n = 50$ vyšlo $p=0.0037$, tedy můžeme nulovou hypotézu zamítnout.\\
Pro pár $n = 20$ a $n = 50$ vyšlo $p=0.9084$, tedy hypotézu zamítnout nelze.

\begin{figure}[h]
    \centering
    \begin{minipage}{0.49\textwidth}
        \centering
        \includegraphics[width=\textwidth]{ks_test_15_50.png}
        \caption{KS test pro $n = 15$ a $n = 50$}
        \label{fig:ks_test_15_50}
    \end{minipage}
    \hfill
    \begin{minipage}{0.49\textwidth}
        \centering
        \includegraphics[width=\textwidth]{ks_test_20_50.png}
        \caption{KS test pro $n = 20$ a $n = 50$}
        \label{fig:ks_test_20_50}
    \end{minipage}
\end{figure}

\subsection*{Závěr a limitace}
Vyšlo nám, že pro dostatečnou náhodnost kostky nestačí 15 náhodných tahů, ale u 20 se nám toto potvrdit nepodařilo.
Je ale důležité zmínit, že není zaručeno, že se dají výsledky zobecnit pro lidské řešitele případně jiné algoritmy.
Například 20 tahů jistě není dostatečný počet pro řešiče hledající nejkratší řešení,
neboť ta mohou být dlouhé i 20 tahů.

\end{document}