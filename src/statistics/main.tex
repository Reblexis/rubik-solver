\documentclass{article}
\usepackage{amsmath}

\title{Zkoumání parametrů řešiče Rubikovy kostky}
\author{Viktor Číhal}
\date{}


\begin{document}
\maketitle

\section*{Popis řešiče}
\subsection*{Algoritmus}
Řešič funguje na základě opakovaného prohledávání stavového prostoru metodou DFS s dodatečným přidáváním náhodných tahů, pokud není nalezen zlepšující
stav kostky. K rozhodnutí o tom, zda je nalezen zlepšující stav, se používá metrika, počítající správné kostičky v kostce. 
Hledání je rozděleno do 2 fází. V první fázi se hledá řešení, které dostane kostku do podgrupy G1 a ve druhé fázi se pomocí menšího počtu
různých tahů hledá již úplné řešení.

\subsection*{Parametry}
Při spouštění lze nastavovat tyto parametry:
\begin{itemize}
    \item Tahy zamíchání kostky
    \item Časový limit pro hledání řešení
    \item Maximální hloubka hledání v první fázi
    \item Maximální hloubka hledání ve druhé fázi (může být větší díky větvícímu faktoru)
    \item Počet náhodných tahů při nenalezení zlepšujícího stavu
\end{itemize}






\end{document}