\documentclass{article}
\usepackage{amsmath}
\usepackage{hyperref}

\title{Zkoumání parametrů řešiče Rubikovy kostky}
\author{Viktor Číhal}
\date{}


\begin{document}
\maketitle
\section*{Úvod}
V této práci jsem se rozhodl zkoumat určité parametry svého řešiče Rubikovy kostky.
Zdrojový kód a notebook s experimenty se dá nalézt na \url{https://github.com/Reblexis/rubik-solver}.

\section*{Popis řešiče}
\subsection*{Algoritmus}
Řešič funguje na základě opakovaného prohledávání stavového prostoru metodou DFS
s dodatečným přidáváním náhodných tahů, pokud není nalezen zlepšující stav kostky.
K rozhodnutí o tom, zda je nalezen zlepšující stav, se používá metrika, počítající
správné kostičky v kostce. Hledání je rozděleno do 2 fází. V první fázi se hledá
řešení, které dostane kostku do podgrupy G1 a ve druhé fázi se pomocí menšího počtu
různých tahů (a tedy i větší hloubky) hledá již úplné řešení.\\
Řešič tedy většinou nenalezne nejkratší řešení.

\subsection*{Parametry}
Při spouštění lze nastavovat tyto parametry:
\begin{itemize}
    \item Počet tahů zamíchání kostky ($n$)
    \item Časový limit pro hledání řešení (v milisekundách) ($t$)
    \item Maximální hloubka hledání v první fázi ($d1$)
    \item Maximální hloubka hledání ve druhé fázi ($d2$)
\end{itemize}

Více informací se nachází v \href{https://github.com/Reblexis/rubik-solver/blob/main/README.md}{README.md} souboru.

\section*{Experimenty}
Nejprve provedeme experimenty s časovým omezením na 1 sekundu. Sice to povede k menšímu počtu úplných
řešení, ale umožní nám rychlejší sběr dat. 

\subsection*{Vliv počtu tahů při míchání na dosažené skóre}
Rozhodl jsem se prozkoumat tento vztah, neboť se hodí vědět kolik stačí tahů při míchání, aby byla kostka
dostatečně náhodná. \\
Skóre je počítáno jako počet správně umístěných kostiček v kostce (tedy na správné pozici a zároveň správně otočených).
Pro každou testovanou hodnotu $n$ spustíme 100 testů a ostatní parametry fixujeme na
$t = 1000$, $d1 = 4$ a $d2 = 5$.




\end{document}